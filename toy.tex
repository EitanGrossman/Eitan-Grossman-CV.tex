\documentclass{article}

\usepackage{color}
\usepackage{xcolor}
\usepackage{natbib}
\usepackage{listings}
\lstset{% general command to set parameter(s)
basicstyle=\small, % print whole listing small
%keywordstyle=\color{black}\bfseries\underbar,
% underlined bold black keywords
%identifierstyle=, % nothing happens
%commentstyle=\color{white}, % white comments
%stringstyle=\ttfamily, % typewriter type for strings
%showstringspaces=false,
language={},
frame={single},
backgroundcolor=\color{yellow!30!}
} % no special string spaces

\usepackage{titling}
\setlength{\droptitle}{-10em}

\usepackage{hyperref}


\author{Todd \&~Eitan}
\title{Learning \LaTeX{} \&~Bibliographies}

\begin{document}

\maketitle

\section*{What we're using}
\begin{enumerate}
    \item A simple article class document
        \begin{lstlisting}
        \documentclass{article}
        \end{lstlisting}
    This begins the \texttt{.tex} document.
    \item The package \texttt{natbib}
        \begin{lstlisting}
        \usepackage{natbib}
        \end{lstlisting}
    This goes in your preamble, where you call packages and define terms.
    \item One of the bibliography styles which comes with \texttt{natbib}
        \begin{lstlisting}
        \bibliographystyle{plainnat}
        \end{lstlisting}
    This goes wherever, though usually right next to the following command...
    \item Your starter \texttt{Bibliography.bib} file
        \begin{lstlisting}
        \bibliography{Bibliography.bib}
        \end{lstlisting}
    This goes where you want the bibliography to be printed.
\end{enumerate}

\section*{How to build your \texttt{.bib} file}
\begin{itemize}
    \item Each entry starts with @$\langle$\textit{type}$\rangle$
        \subitem Common types include \textit{article}, \textit{book}, \textit{incollection}, \textit{inproceedings}\textellipsis
    \item Immediately after the type is the key, followed by a comma
        \subitem This is the label you will use to cite this entry
            \begin{lstlisting}
            @phdthesis{snider-2017,
            \end{lstlisting}
        \subitem Note that the bracket isn't closed until the end of the entry.
    \item Then come the fields
    \begin{itemize}
        \item Each entry type has a set of mandatory fields
        \item Each entry type has some optional fields
        \item You can see which is which here:\\ \url{https://en.wikipedia.org/wiki/BibTeX#Entry_types}
    \end{itemize}
    \item Each field has a name, equals, and then its argument (enclosed in curly brackets or quotation marks)\footnote{I strongly recommend brackets over quotation marks. As you get more and more comfortable with \LaTeX, you'll start to differentiate ` and `` from ' and '', and it seeps...}
    \begin{lstlisting}
        year = {2014},
    \end{lstlisting}
        \subitem Every non-final field line has to end in a comma.
\end{itemize}

\section*{How to cite}

The simplest citation is just \texttt{\textbackslash cite}. You can generate\\
\cite{ALH2014}\\
by typing the following:
\begin{lstlisting}
    \cite{ALH2014}
\end{lstlisting}

\vspace{\baselineskip}
\noindent There are other ways, as well. Here is a comparison:\\[12pt]
\begin{tabular}{l l }
\textbackslash \texttt{citet}$\{\textit{key}\}$ & 	Jones et al. (1990) \\
\textbackslash \texttt{citet*}$\{\textit{key}\}$ & 	Jones, Baker, and Smith (1990) \\
\textbackslash \texttt{citep}$\{\textit{key}\}$	& (Jones et al. 1990) \\
\textbackslash \texttt{citep*}$\{\textit{key}\}$ & 	(Jones, Baker, and Smith 1990) \\
\textbackslash \texttt{citep[p.~99]}$\{\textit{key}\}$ & 	(Jones et al., 1990, p. 99) \\
\textbackslash \texttt{citep[e.g.][]}$\{\textit{key}\}$ & 	(e.g. Jones et al., 1990) \\
\textbackslash \texttt{citep[e.g.][p.~99]}$\{\textit{key}\}$ & 	(e.g. Jones et al., 1990, p. 99) \\
\textbackslash \texttt{citeauthor}$\{\textit{key}\}$ &	Jones et al. \\
\textbackslash \texttt{citeauthor*}$\{\textit{key}\}$ &	Jones, Baker, and Smith \\
\textbackslash \texttt{citeyear}$\{\textit{key}\}$ &	1990 \\
\textbackslash \texttt{citeapos}$\{\textit{key}\}$\texttt{*} &	Jones et al.'s (1990)
\end{tabular}
\vspace{\baselineskip}

*Assumes \textbackslash citeapos is defined in your style or document like this:
\begin{lstlisting}
    \def\citeapos#1{\citeauthor{#1}'s (\citeyear{#1})}
\end{lstlisting}

\noindent
You can change whether these show up in parentheses or square brackets, with numbers or names, etc., by either
\begin{itemize}
    \item changing the bibliography style, or
    \item using a custom \texttt{\textbackslash bibpunct} command, e.g.:
    \begin{lstlisting}
        \bibpunct{[}{]}{,}{a}{}{;}
    \end{lstlisting}
\end{itemize}

\noindent Changing the bibliography style also changes how the reference list is formatted.

\bibliographystyle{plainnat}
\bibliography{Bibliography.bib}

\end{document}